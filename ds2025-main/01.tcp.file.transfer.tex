\documentclass{article}
\usepackage{graphicx}
\usepackage{listings}
\usepackage{color}

\title{TCP File Transfer – Practical Work 1}
\author{Nguyen Hoang Long - Distributed Systems}

\begin{document}
\maketitle

\section{Introduction}

This practical work aims to develop a simple file transfer system using TCP/IP in a command-line interface. The system includes one server and one client using sockets.

\section{Protocol Design}

Our protocol is based on a simple request–response model:

\begin{enumerate}
    \item Client connects to server.
    \item Client sends filename.
    \item Client sends file size.
    \item Server replies \texttt{READY}.
    \item Client sends file data in binary chunks.
    \item Server stores the file and responds \texttt{OK}.
\end{enumerate}

Figure~\ref{fig:proto} illustrates the protocol flow.

\begin{verbatim}
Client                                   Server
  |                                         |
  | ---- Connect via TCP -----------------> |
  | ---- Send filename -------------------> |
  | ---- Send filesize -------------------> |
  | <------------- READY ------------------ |
  | ---- Send file data (chunks) ---------> |
  | <----------------- OK ----------------- |
  | ---- Close connection ----------------> |
\end{verbatim}

\section{System Organization}

Figure~\ref{fig:system} shows the architecture.

\begin{verbatim}
+------------------+                +-------------------+
|      Client      |   TCP/IP       |       Server      |
|------------------| <----------->  |-------------------|
| - CLI interface  |                | - Listener socket |
| - File reader    |                | - File writer     |
| - TCP socket     |                | - TCP socket      |
+------------------+                +-------------------+
\end{verbatim}

\section{Implementation}

\subsection{Server Code}

\begin{lstlisting}[language=Python]
import socket

HOST = "0.0.0.0"
PORT = 5001

server = socket.socket(socket.AF_INET, socket.SOCK_STREAM)
server.bind((HOST, PORT))
server.listen(1)

conn, addr = server.accept()

filename = conn.recv(1024).decode().strip()
filesize = int(conn.recv(1024).decode().strip())

conn.send(b"READY")

with open(filename, "wb") as f:
    received = 0
    while received < filesize:
        data = conn.recv(4096)
        if not data:
            break
        f.write(data)
        received += len(data)

conn.send(b"OK")
conn.close()
\end{lstlisting}

\subsection{Client Code}

\begin{lstlisting}[language=Python]
import socket
import os

HOST = "127.0.0.1"
PORT = 5001

filename = "test.txt"
filesize = os.path.getsize(filename)

client = socket.socket(socket.AF_INET, socket.SOCK_STREAM)
client.connect((HOST, PORT))

client.send((filename + "\n").encode())
client.send((str(filesize) + "\n").encode())

if client.recv(1024).decode() != "READY":
    client.close()

with open(filename, "rb") as f:
    data = f.read(4096)
    while data:
        client.send(data)
        data = f.read(4096)

print(client.recv(1024).decode())
client.close()
\end{lstlisting}

\section{Work Distribution}

\begin{itemize}
    \item Member A: Protocol design, server implementation
    \item Member B: Client implementation, testing
    \item Member C: LaTeX report, diagrams
\end{itemize}

\section{Conclusion}

We successfully implemented a TCP file transfer system using one client and one server. The protocol works reliably for text and binary files. The system demonstrates basic distributed communication using sockets.

\end{document}
